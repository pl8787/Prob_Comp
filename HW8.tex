\documentclass[12pt]{article}
\usepackage[ansinew]{inputenc} % ASCII (Western CP)
\usepackage{graphicx}
\usepackage{color}
\usepackage[colorlinks]{hyperref}
\usepackage{geometry}
\usepackage{amsmath}
\usepackage{amsfonts}
\geometry{left=2.5cm,right=2.5cm,top=2.5cm,bottom=2.5cm}

\title{Probability and Computing - HW8}
\author{Pang Liang\\ Student No. 201418013229033}

\begin{document}
\maketitle

\section{Problem1}
Assume that two states $i$ and $j$ of a Markov chain belong to the same communicating class. Prove that $i$ is recurrent if so is $j$. [Hint: you can prove either straightforwardly or by showing that $i$ is recurrent if and only if $\sum_n p_{ii}^{(n)} = +\infty$. This formula follows from the fact that $p_{ii}^{(n)} = \sum_{n=1}^n r_{ii}^{(m)} p_{ii}^{(n-m)} $, where $r_{ii}^{(m)}$ is the probability that the chain first returns to $i$ at step $m$ if it starts at state $i$].

Solution:\\

Suppose that state $i$ is recurrent, so $\sum_n p_{ii}^{(n)} = +\infty$. As we know state $j$ and $i$ belong to the same communicating class. So there exist $k_1$ and $k_2$ that $p_{ij}^{(k_1)} >0$ and $p_{ji}^{(k_2)} >0$.

For all $n>0$ that
\begin{equation}
    p_{jj}^{(k_1+k_2+n)} \ge p_{ji}^{(k_1)} p_{ii}^{(n)} p_{ij}^{(k_2)}
\end{equation}
so we have
\begin{equation}
    \begin{split}
    \sum_n p_{jj}^{(n)} &\ge \sum_n p_{jj}^{(k_1+k_2+n)} \\
    &\ge \sum_n p_{ji}^{(k_1)} p_{ii}^{(n)} p_{ij}^{(k_2)} \\
    &= p_{ji}^{(k_1)} p_{ij}^{(k_2)} \sum_n p_{ii}^{(n)} \\
    &\ge +\infty
    \end{split}
\end{equation}
So $j$ is recurrent.

\section{Problem2}
Assume that two states $i$ and $j$ of a Markov chain belong to the same communicating class and are recurrent. It is known that $i$ is null recurrent if and only if the multi-step transition probability $p_{ii}^{(n)} $ satisfies $\lim_{n \to +\infty}p_{ii}^{(n)} = 0 $. Prove that $j$ is null recurrent if so is $i$.

Solution:\\

Suppose that state $j$ is recurrent, so $\lim_{n \to +\infty}p_{jj}^{(n)} = 0 $. As we know state $j$ and $i$ belong to the same communicating class. So there exist $k_1$ and $k_2$ that $p_{ij}^{(k_1)} >0$ and $p_{ji}^{(k_2)} >0$.

For all $n>0$ that
\begin{equation}
    \begin{split}
    p_{jj}^{(k_1+k_2+n)} &\ge p_{ji}^{(k_1)} p_{ii}^{(n)} p_{ij}^{(k_2)}\\
    p_{ii}^{(n)} &\le p_{ji}^{(k_1)} p_{jj}^{(k_1+k_2+n)} p_{ij}^{(k_2)}
    \end{split}
\end{equation}
so we have
\begin{equation}
    \begin{split}
    \lim_{n \to +\infty}p_{ii}^{(n)} &\le \lim_{n \to +\infty} p_{ji}^{(k_1)} p_{jj}^{(k_1+k_2+n)} p_{ij}^{(k_2)} \\
    &\le p_{ji}^{(k_1)} \lim_{n \to +\infty}p_{jj}^{(k_1+k_2+n)} p_{ij}^{(k_2)} \\
    &\le 0
    \end{split}
\end{equation}

Because $\lim_{n \to +\infty}p_{ii}^{(n)} \ge 0$ ,so $\lim_{n \to +\infty}p_{ii}^{(n)} = 0$. $j$  is null recurrent.

\section{Problem3}
Given a finite markov chain, where finiteness means that there are a finite number of states, prove that
\begin{enumerate}
\item At least one state is recurrent.
\item All recurrent states are positive recurrent.
\end{enumerate}

Solution:\\
\begin{enumerate}
    \item
     Since there are a finite number of communicating classes, and
since once the chain leaves a communicating class it cannot return, it must
eventually settle into one communicating class. Thus, at least one state in this
class is visited an unbounded number of times after the chain enters it.
    \item
    Let $C(i)$ denote the communicating contains state $i$. Let $p$ be the largest transition probability less than 1 from any states in $C(i)$.
    \begin{equation}
        h_{i,i} = \sum_t t r_{i,i}^t \le \sum_t t p^t
    \end{equation}
    Since there are finite state so $r_{i,i}^t = 0, t>n$. So 
    \begin{equation}
        h_{i,i} \le \sum_{t=1}^{n} t p^t \le \frac{n}{1-p} < \infty
    \end{equation}
    Then $i$ is positive recurrent.

\end{enumerate}


\section{Problem4}
Do Bernoulli experiment for 20 trials, using a new 1-Yuan coin. Record the result in a
string $s_1s_2 \cdots s_i \cdots s_{20}$, where $s_i$ is 1 if the $i^{th}$ trial gets Head, and otherwise is 0.

0101011001 0011101000

\end{document}
